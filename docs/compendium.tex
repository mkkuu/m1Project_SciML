\documentclass[a4paper,11pt]{article}
\usepackage[utf8]{inputenc}
\usepackage[T1]{fontenc}
\usepackage[french]{babel}
\usepackage{lmodern}
\usepackage{geometry}
\usepackage{graphicx}
\usepackage{tikz}
\usepackage[colorlinks=true,linkcolor=blue,citecolor=blue]{hyperref}
\usepackage[acronym]{glossaries}


% Mise en Page
\geometry{margin=2.5cm}

% Graphs et Schémas
\usetikzlibrary{arrows.meta,positioning}

% Glossaire
\makeglossaries

% Fichier avec définitions
% Contient les définitions pour le package glossaries

% -------- Liste d'acronymes --------
\newacronym{sst}{SST}{Sea Surface Temperature (température de surface de la mer)}

% -------- Termes généraux --------
\newglossaryentry{climatologie}{
    name={climatologie},
    description={Moyenne statistique d'une grandeur
    calculée sur une période de référence, utilisée
    comme état moyen de comparaison}
}




% ======== INFOS DOC ========
\title{Compendium océanographique pour l'analyse de la SST en Manche (2010--2020)}
\author{DELAUNAY Robin, HENRY Williams}
\date{\today}

\begin{document}

\maketitle

\tableofcontents
\newpage

% ======== PART 1 : INTRO ========
\section{Introduction}

Ce document document a pour but de rassembler des éléments de contexte océanographique,
 les principaux phénomènes climatiques et les définitions des termes techniques utilisés
 dans le cadre du projet de prédiction de la température à la surface de la Manche sur la
 période 2010--2020.

 Il complète le rapport en apportant :
 \begin{itemize}
    \item une mise en contexte régionale de la Manche
    \item une frise chronologique qualitative de l'évolution de la SST sur 2010--2020
    \item un focus sur la rupture temporelle détectée autour de 2014
    \item un glossaire des termes océanographiques et statistiques mobilisé avec des
    schémas simples.
 \end{itemize}

 % ======== 2.Contexte régionale ========

 \section{Contexte océanographique de la Manche}

 \subsection{Cadre géographique et caractéristiques générales}

 \subsection{Rôle climatique et dynamique générale}

 % ======== 3.Frise Chronologique ========
 \section{Evolution qualitative de la SST en Manche (2010--2020)}

 Cette section propose une frise chronologique qualitative de l'évolution
 de la SST en Manche sur la période 2010--2020. Il s'agit d'un schéma conceptuel
 destiné à résumer les grands types de régimes thermiques identifiés dans le projet
 (périodes proches de la climatologie, régime plus chaud, etc...).

 \begin{figure}[ht]
  \centering
  \begin{tikzpicture}[
    timeline/.style={thick},
    tick/.style={thin},
    event/.style={circle, fill=black, inner sep=1.2pt},
    label/.style={font=\small, align=center}
  ]

    % Ligne de temps principale
    \draw[timeline] (0,0) -- (10,0);
    
    % Repères annuels (1 unité = 1 an)
    \foreach \x/\year in {
      0/2010,1/2011,2/2012,3/2013,4/2014,
      5/2015,6/2016,7/2017,8/2018,9/2019,10/2020} {
      \draw[tick] (\x,0.1) -- (\x,-0.1)
        node[below=2pt,font=\scriptsize] {\year};
    }

    % Exemple d'événement : rupture temporelle autour de 2014
    \node[event] (e2014) at (4,0) {};
    \node[label,above=4pt of e2014] {Rupture\\temporelle};

    % Exemple de période plus chaude après 2014 (à adapter)
    \draw[thick] (4.2,0.25) -- (8.8,0.25);
    \node[label,above=4pt] at (6.5,0.25) {Régime thermique\\plus chaud};

    % Exemple de période proche de la climatologie avant 2013 (à adapter)
    \draw[thick] (0.2,-0.25) -- (2.8,-0.25);
    \node[label,below=4pt] at (1.5,-0.25) {SST proche\\de la climatologie};

  \end{tikzpicture}
  \caption{Frise chronologique qualitative de l'évolution de la SST en Manche sur la période 2010--2020 (schéma conceptuel à adapter en fonction des résultats).}
  \label{fig:frise-2010-2020}
\end{figure}

% ======== 4.Rupture temporelle de 2014 ========

\section{Rupture temporelle détectée autour de 2014}

Dans cette section, on décrit la rupture temporelle détectée autour de 2014 dans la série de SST dans la Manche.

% ======== 5.Glossaire des temps et phénomènes ========

\section{GLossaire des termes et phénomènes}

Le glossaire suivant rassemble les acronymes et les termes techniques utilisés dans ce Compendium
et dans le rapport principal.

\subsection{Liste des acronymes}

\printglossary[type=\acronymtype,title={Liste des acronymes}]
\gls{sst} BLABLABLA
\subsection{Glossaire des termes}

\printglossary[title={Glossaire des termes océanographiques et statistiques}]
\gls{climatologie}
\end{document}


