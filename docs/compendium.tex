\documentclass[a4paper,11pt]{article}
\usepackage[utf8]{inputenc}
\usepackage[T1]{fontenc}
\usepackage[french]{babel}
\usepackage{lmodern}
\usepackage{geometry}
\usepackage{graphicx}
\usepackage{tikz}
\usepackage[colorlinks=true,linkcolor=blue,citecolor=blue]{hyperref}
\usepackage[acronym]{glossaries}
\usepackage{subcaption}


% Mise en Page
\geometry{margin=2.5cm}

% Graphs et Schémas
\usetikzlibrary{arrows.meta,positioning}

% Glossaire
\makeglossaries

% Fichier avec définitions
% Contient les définitions pour le package glossaries

% -------- Liste d'acronymes --------
\newacronym{sst}{SST}{Sea Surface Temperature (température de surface de la mer)}

% -------- Termes généraux --------
\newglossaryentry{climatologie}{
    name={climatologie},
    description={Moyenne statistique d'une grandeur
    calculée sur une période de référence, utilisée
    comme état moyen de comparaison}
}




% ======== INFOS DOC ========
\title{Compendium océanographique pour l'analyse de la SST en Manche (2010--2020)}
\author{DELAUNAY Robin, HENRY Williams}
\date{\today}

\begin{document}

\maketitle

\tableofcontents
\newpage

% ======== PART 1 : INTRO ========
\section{Introduction}

Ce document document a pour but de rassembler des éléments de contexte océanographique,
 les principaux phénomènes climatiques et les définitions des termes techniques utilisés
 dans le cadre du projet de prédiction de la température à la surface de la Manche sur la
 période 2010--2020.

 Il complète le rapport en apportant :
 \begin{itemize}
    \item une mise en contexte régionale de la Manche
    \item une frise chronologique qualitative de l'évolution de la SST sur 2010--2020
    \item un focus sur la rupture temporelle détectée autour de 2014
    \item un glossaire des termes océanographiques et statistiques mobilisé avec des
    schémas simples.
 \end{itemize}

 % ======== 2.Contexte régionale ========

 \section{Contexte océanographique de la Manche}

 \subsection{Cadre géographique et caractéristiques générales}
 \begin{figure}[ht]
  \centering
  \begin{subfigure}[t]{0.48\textwidth}
    \centering
    \includegraphics[height=6cm]{referenceMap.png}
    \caption{Carte de contexte géographique de la Manche.}
  \end{subfigure}
  \hfill
  \begin{subfigure}[t]{0.48\textwidth}
    \centering
    \includegraphics[height=6cm]{mapTemporalDeviations.png}
    \caption{Écart-type temporel de la SST (2010--2020).}
  \end{subfigure}
  \caption{Comparaison entre la carte de contexte géographique et la carte de
  variabilité temporelle de la SST en Manche.}
\end{figure}


\subsection{Rôle climatique et dynamique générale}


 % ======== 3.Frise Chronologique ========
 \section{Evolution qualitative de la SST en Manche (2010--2020)}

 Cette section propose une frise chronologique qualitative de l'évolution
 de la SST en Manche sur la période 2010--2020. Il s'agit d'un schéma conceptuel
 destiné à résumer les grands types de régimes thermiques identifiés dans le projet
 (périodes proches de la climatologie, régime plus chaud, etc...).

 \begin{figure}[ht]
  \centering
  \begin{tikzpicture}[
    timeline/.style={thick},
    tick/.style={thin},
    event/.style={circle, fill=black, inner sep=1.2pt},
    label/.style={font=\small, align=center}
  ]

    % Ligne de temps principale
    \draw[timeline] (0,0) -- (10,0);
    
    % Repères annuels (1 unité = 1 an)
    \foreach \x/\year in {
      0/2010,1/2011,2/2012,3/2013,4/2014,
      5/2015,6/2016,7/2017,8/2018,9/2019,10/2020} {
      \draw[tick] (\x,0.1) -- (\x,-0.1)
        node[below=2pt,font=\scriptsize] {\year};
    }

    \node[event] (e2010) at (0,0) {};
    \node[label,above=4pt of e2010] {ENSO\\froide};

    \node[event] (e2014) at (4,0) {};
    \node[label,above=4pt of e2014] {Rupture\\temporelle};
    
  
    \draw[thick] (5,0.25) -- (6,0.25);
    \node[label,above=4pt] at (6.5,0.25) {El Niño\\plus important};

    
    \draw[thick] (1,-0.25) -- (2,-0.25);
    \node[label,below=4pt] at (1.5,-0.25) {Etat thermique bas};

    \draw[thick] (5,-0.25) -- (10,-0.25);
    \node[label,below=4pt] at (7.5,-0.25) {Etat thermique haut};

  \end{tikzpicture}
  \caption{Frise chronologique qualitative de l'évolution de la SST en Manche sur la période 2010--2020.}
  \label{fig:frise-2010-2020}
\end{figure}

% ======== 4.Rupture temporelle de 2014 ========

\section{Rupture temporelle détectée autour de 2014}

Dans cette section, on décrit la rupture temporelle détectée autour de 2014 dans la série de SST dans la Manche.
La rupture temporelle détectée en 2014 marque un changement de régime thermique du système océan–atmosphère. 
Avant 2014, la dynamique est dominée par des forçages climatiques froids, notamment un ENSO froid et plusieurs 
phases de NAO négative, entraînant des pertes énergétiques hivernales suffisantes pour limiter le réchauffement 
de surface. Malgré cela, un réchauffement de fond progressif est déjà présent, traduisant l’accumulation lente 
de chaleur dans l’océan.

À partir de 2014, ce réchauffement de fond élève le plancher thermique moyen du système. 
Les mécanismes atmosphériques, bien que toujours actifs, ne produisent plus des anomalies 
froides suffisantes pour compenser les gains énergétiques. Le système bascule alors vers un 
nouvel état d’équilibre caractérisé par une SST moyenne plus élevée. Cette transition est 
cohérente avec l’augmentation observée des moyennes et des quantiles de température après 
2014, confirmant un changement de régime plutôt qu’une simple fluctuation interannuelle.
% ======== 5.Glossaire des temps et phénomènes ========

\section{GLossaire des termes et phénomènes}

Le glossaire suivant rassemble les acronymes et les termes techniques utilisés dans ce Compendium
et dans le rapport principal.

\subsection{Liste des acronymes}

\printglossary[type=\acronymtype,title={Liste des acronymes}]
\gls{sst}
\gls{nao}
\gls{enso}
\gls{en}
\gls{ln}
\gls{oa}
\gls{vi}

\subsection{Glossaire des termes}

\printglossary[title={Glossaire des termes océanographiques et statistiques}]
\gls{climatologie}
\gls{north atlantic oscillation}
\gls{el niño-southern oscillation}
\gls{stratification}
\gls{refroidissement atmosphérique}
\gls{mélange vertical}
\gls{forçages atmosphériques}

\end{document}


