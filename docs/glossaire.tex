% Contient les définitions pour le package glossaries

% -------- Liste d'acronymes --------
\newacronym{sst}{SST}{Sea Surface Temperature (température de surface de la mer)}
\newacronym{nao}{NAO}{Oscillation nord-atlantique}
\newacronym{enso}{ENSO}{El Niño-Southern Oscillation}
\newacronym{en}{EN}{El Niño}
\newacronym{ln}{LN}{La Niña}
\newacronym{oa}{OA}{Ocean-Atmosphere system}
\newacronym{vi}{VI}{Interannual Variability}
% -------- Termes généraux --------
\newglossaryentry{climatologie}{
    name={climatologie},
    description={Moyenne statistique d'une grandeur
    calculée sur une période de référence, utilisée
    comme état moyen de comparaison}
}

\newglossaryentry{stratification}{
    name={stratification},
    description={Organisation verticale de l’océan 
    en couches de densité différente, principalement 
    contrôlée par la température et la salinité. 
    Une stratification forte limite les échanges 
    verticaux de chaleur et de matière entre la surface
     et les couches profondes.}
}

\newglossaryentry{refroidissement atmosphérique}{
    name={refroidissement atmosphérique},
    description={Processus par lequel l’atmosphère 
    induit une perte de chaleur de l’océan, notamment 
    par les flux turbulents (flux sensible et latent), 
    le rayonnement infrarouge et l’advection d’air froid, 
    particulièrement efficace durant les périodes hivernales.}
}

\newglossaryentry{mélange vertical}{
    name={mélange vertical},
    description={Processus dynamique 
    assurant le transfert de chaleur, 
    de sel et de nutriments entre les 
    différentes couches de l’océan, sous 
    l’effet du vent, de la convection thermique 
    et de l’instabilité de la colonne d’eau. Le 
    mélange vertical tend à homogénéiser la 
    température sur une certaine profondeur.}
}

\newglossaryentry{forçages atmosphériques}{
    name={forçages atmosphériques},
    description={Ensemble des mécanismes par 
    lesquels l’atmosphère influence l’océan, 
    notamment via les flux de chaleur 
    (sensible et latent), le rayonnement, le 
    vent et les conditions de pression. 
    Les forçages atmosphériques contrôlent une 
    grande partie de la variabilité de la température 
    de surface de la mer à des échelles saisonnières à 
    interannuelles.}
}

\newglossaryentry{north atlantic oscillation}{
    name={north atlantic oscillation},
    description={Mode dominant de variabilité atmosphérique 
    sur l’Atlantique Nord, défini par la différence de pression 
    entre l’anticyclone des Açores et la dépression d’Islande, 
    qui module l’intensité des vents d’ouest et les conditions 
    hivernales en Europe}
}

\newglossaryentry{el niño-southern oscillation}{
    name={el niño-southern oscillation},
    description={Oscillation climatique couplée océan–atmosphère 
    dans le Pacifique tropical, caractérisée par des anomalies de 
    température de surface de la mer et de circulation atmosphérique. 
    ENSO alterne entre trois phases principales : El Niño, La Niña et 
    une phase neutre, et exerce une influence globale sur la circulation 
    atmosphérique et les températures océaniques.}
}